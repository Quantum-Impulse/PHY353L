\documentclass[12pt]{article}

\usepackage{graphicx} % For including graphics
\usepackage{amsmath} % For math formatting
\usepackage{geometry} % For page layout
\geometry{a4paper, margin=1in}

\title{Effect of Lead Thickness on Radiation Detection}
\author{Your Name\\
        Department of Physics, Your University}
\date{\today}

\begin{document}

\maketitle

\begin{abstract}
    Briefly summarize the purpose of the experiment, the main findings, and the conclusions drawn.
\end{abstract}

\section{Introduction}
    Introduce the concept of radioactive decay, the significance of lead as a shielding material, and the objectives of your experiment.

\section{Methodology}
    Describe the experimental setup, the radioactive source used, the range of lead thicknesses tested, and the procedure followed during the experiment.

\section{Results}
    Present the data obtained from the experiment. Include your Python-generated graphs here. Use the following syntax to include images:
    
    \begin{figure}[h!]
        \centering
        \includegraphics[width=0.8\textwidth]{path/to/your/graph.png}
        \caption{Caption describing the graph}
        \label{fig:graph1}
    \end{figure}

\section{Discussion}
    Analyze the results, discussing how different thicknesses of lead affected the detection of radiation. Compare your findings with theoretical expectations and previous studies.

\section{Conclusion}
    Summarize the main findings of the experiment, the implications of these findings, and potential areas for further research.

\section{References}
    List the references cited in your report.

\end{document}
