\documentclass[12pt]{article}

\usepackage{graphicx} % For including graphics
\usepackage{amsmath} % For math formatting
\usepackage{geometry} % For page layout
\geometry{a4paper, margin=1in}

\title{Effect of Lead Thickness on Radiation Detection}
\author{Enrique Rivera Jr. \\
        Physics Undergraduate, \\ 
        The University of Texas at Austin}
\date{\today}

\begin{document}

\maketitle

\begin{abstract}
    In our investigation, we examined how varying thicknesses of lead 
    influence the detection of radiation emitted by Cobalt-60, a widely utilized radioactive isotope. 
    Utilizing a scintillator, we quantified the radiation intensity across different lead shields. 
    Our observations revealed a predictable decline in radiation intensity with increased lead thickness, 
    aligning with theoretical expectations. Notably, this attenuation follows an exponential trend, 
    underscoring the complex dynamics of radiation shielding. These insights contribute valuable knowledge towards 
    optimizing radiation protection materials, emphasizing the exponential nature of radiation attenuation 
    through lead. This understanding is crucial for designing more effective shielding strategies in 
    both medical and industrial applications, enhancing safety protocols against radioactive exposure.
\end{abstract}

\section{Introduction}
    The study of radioactive decay remains a cornerstone in understanding nuclear processes, 
    with Cobalt-60 frequently serving as a pivotal subject due to its prevalent use in both 
    medical and industrial applications. This isotope's emission characteristics offer a 
    valuable window into the dynamics of gamma radiation, a form of energy crucial in a 
    variety of fields. The attenuation of this radiation by shielding materials, particularly lead, 
    is of paramount importance for ensuring safety and efficacy in these applications. 
    Lead is traditionally favored for its high density and atomic number, which contribute to 
    its effectiveness in absorbing gamma radiation. However, the relationship between lead thickness 
    and radiation attenuation is not merely a matter of linear correlation but involves more 
    complex interactions that significantly influence shielding design. This experiment aims to elucidate 
    these interactions by systematically varying the thickness of lead shielding to observe its impact 
    on radiation detection from Cobalt-60, thereby providing insights that could refine our 
    approach to radiation protection.
\section{Experimental Setup and Procedure}
    \subsection{Materials and Equipment}
        The primary materials and equipment used in this experiment include:
        \begin{itemize}
            \item Cobalt-60 radioactive source
            \item Scintillator
            \item Lead sheets of varying thicknesses
            \item Data acquisition system (Maestro)
            \item Python programming environment (miniconda with Python Notebooks)
        \end{itemize}
        
    \subsection{Apparatus}
        The experimental setup consists of a Cobalt-60 source, a scintillator, and lead sheets of varying thicknesses. 
        The Cobalt-60 source emits gamma radiation, which is detected by the scintillator. 
        The lead sheets are placed between the source and the scintillator to measure the attenuation of the radiation. 
        The scintillator is connected to a data acquisition system, which records the intensity of the radiation over time. 
        The data acquisition system is controlled by a Python program, which allows us to measure the radiation intensity at different lead thicknesses.

    \subsection{Scintillator}
        The scintillator is a device that detects radiation by converting the energy of incoming photons into visible light. 
        This light is then detected by a photomultiplier tube, which amplifies the signal and converts it into an electrical pulse. 
        The scintillator used in this experiment is a sodium iodide (NaI) crystal, which is commonly used for detecting gamma radiation. 
        The crystal is coupled to a photomultiplier tube, which amplifies the light signal and converts it into an electrical pulse. 
        The electrical pulse is then processed by a data acquisition system, which records the number of pulses over a given time interval. 
        This allows us to measure the intensity of the radiation emitted by the Cobalt-60 source.

    \subsection{Photomultiplier to Electronic Multiplier}
        The photomultiplier tube is a device that converts the light signal from the scintillator into an electrical pulse. 
        It consists of a series of dynodes, which are metal electrodes that are held at successively higher voltages. 
        When a photon strikes the first dynode, it releases an electron, which is then accelerated towards the next dynode. 
        This process is repeated at each dynode, resulting in a cascade of electrons that is amplified at each stage. 
        The final output is a large number of electrons, which is then converted into an electrical pulse. 
        This pulse is then processed by the data acquisition system, which records the number of pulses over a given time interval. 
        This allows us to measure the intensity of the radiation emitted by the Cobalt-60 source.
        
    \subsection{Single and Multi-Channel Analyzer (Cobalt-60 Gamma Ray Spectrum)}
        The single-channel analyzer (SCA) is a device that allows us to select a specific range of pulse heights from the photomultiplier tube. 
        This is useful for filtering out background radiation and other unwanted signals. 
        The multi-channel analyzer (MCA) is a device that allows us to record the number of pulses at each pulse height. 
        This allows us to measure the intensity of the radiation emitted by the Cobalt-60 source as a function of pulse height. 
        The MCA is connected to a computer, which allows us to record and analyze the data. 
        The data is then processed using a Python program, which allows us to measure the radiation intensity at different lead thicknesses.
        
    \subsection{Data Collection}
        The data acquisition system records the intensity of the radiation emitted by the Cobalt-60 source at different lead thicknesses. 
        The data is then processed using a Python program, which allows us to measure the radiation intensity at each thickness. 
        This data is then used to generate graphs of the radiation intensity as a function of lead thickness. 
        These graphs are then analyzed to determine the relationship between lead thickness and radiation attenuation.
        

\section{Results}
    Present the data obtained from the experiment. Include your Python-generated graphs here. Use the following syntax to include images:
    
    \begin{figure}[h!]
        \centering
        \includegraphics[width=0.8\textwidth]{path/to/your/graph.png}
        \caption{Caption describing the graph}
        \label{fig:graph1}
    \end{figure}

\section{Conclusion}
    Summarize the main findings of the experiment, the implications of these findings, and potential areas for further research.

\section{References}
    List the references cited in your report.

\end{document}
