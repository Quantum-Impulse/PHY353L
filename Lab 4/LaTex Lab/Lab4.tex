\documentclass[12pt]{article}

\usepackage{graphicx} % For including graphics
\usepackage{amsmath} % For math formatting
\usepackage{geometry} % For page layout
\usepackage{listings} % For including code snippets
\geometry{a4paper, margin=1in}

\title{The Photoelectric Effect}
\author{Enrique Rivera Jr. \\
                Physics Undergraduate, \\ 
                The University of Texas at Austin}
\date{\today}

\begin{document}
\maketitle

\begin{abstract}

\end{abstract}

\section{Introduction}
        \subsection{Background and Theory}



\section{Experimental Setup and Procedure}
        \subsection{Light Box}
                

\section{Results}
        \subsection{Data and Analysis}



\section{Conclusion}

\section{References}
    \begin{enumerate}
        \sloppy

        \item  T. Matsumoto. A chaotic attractor from Chua’s circuit. IEEE Trans. Circuits Sys., 31(12):1055–1058, 1984.
        \item  http://www.chuacircuits.com For more information on setup, examples, and matlab example code.
        \item L. O. Chua, C. W. Wu, A. Huang and Guo-Qun Zhong, "A universal circuit for studying and generating chaos. I. Routes to chaos," in IEEE Transactions on Circuits and Systems I: Fundamental Theory and Applications, vol. 40, no. 10, pp. 732-744, Oct. 1993, doi: 10.1109/81.246149.
        \item Jihua Yang, Liqin Zhao,
        Bifurcation analysis and chaos control of the modified Chua’s circuit system,
        Chaos, Solitons and Fractals,
        Volume 77,
        2015,
        Pages 332-339,
        ISSN 0960-0779,
        https://doi.org/10.1016/j.chaos.2015.05.028.
        (https://www.sciencedirect.com/science/article/pii/S0960077915001642)

        

    \end{enumerate}

\end{document}
