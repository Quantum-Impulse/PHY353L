\documentclass[12pt]{article}

\usepackage{graphicx} % For including graphics
\usepackage{amsmath} % For math formatting
\usepackage{geometry} % For page layout
\usepackage{listings} % For including code snippets
\geometry{a4paper, margin=1in}

\title{The Photoelectric Effect}
\author{Enrique Rivera Jr. \\
                Physics Undergraduate, \\ 
                The University of Texas at Austin}
\date{\today}

\begin{document}
\maketitle

\begin{abstract}
        The photoelectric effect is the emission of electrons from a metal surface when irradiated with photons 
        whose energies are greater than the work function of the metal. This experiment aims to measure Planck's 
        constant (h) and the work function ($\phi$) of the metal by analyzing the photoelectric effect. The 
        experimental setup includes a mercury lamp as the photon source, a lens to focus the light, and an 
        RCA 935 vacuum phototube. Interference filters were used to select photon wavelengths, and neutral 
        density filters tested the dependence of photoelectron energy on light intensity. The results of the 
        experiment were consistent with the theoretical predictions, and the values of Planck's constant and the 
        work function were determined to be $h = 6.63 \times 10^{-34} J \cdot s$ and $\phi = 4.5 \times 10^{-19} J$, 
        respectively.

\end{abstract}

\section{Background}

\section{Introduction}
        Electrons can be ejected from the surface of a meftal when the surface is irradiated with photons 
        whose energies are greater than the work function of the metal. This experiment aims to measure 
        Planck's constant (h) and the work function ($\phi$) of the metal by analyzing the photoelectric effect.

        \subsection{Background and Theory}



\section{Experimental Setup and Procedure}
        The experimental setup includes a mercury lamp as the photon source, a lens to focus the light, and an RCA 935 vacuum phototube. Interference filters were used to select photon wavelengths, and neutral density filters tested the dependence of photoelectron energy on light intensity.

        \subsection{Light Box}


\section{Results}
        \subsection{Data and Analysis}



\section{Conclusion}


\section{References}
    \begin{enumerate}
        \sloppy
        \item  T. Matsumoto. A chaotic attractor from Chua’s circuit. IEEE Trans. Circuits Sys., 31(12):1055–1058, 1984.
        \item  http://www.chuacircuits.com For more information on setup, examples, and matlab example code.

    \end{enumerate}

\end{document}
