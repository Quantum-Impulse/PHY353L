\documentclass[12pt]{article}

\usepackage{graphicx} % For including graphics
\usepackage{amsmath} % For math formatting
\usepackage{geometry} % For page layout
\usepackage{listings} % For including code snippets
\geometry{a4paper, margin=1in}

\title{Speed of light}
\author{Enrique Rivera Jr. \\
                Physics Undergraduate, \\ 
                The University of Texas at Austin}
\date{\today}

\begin{document}
\maketitle

\begin{abstract}        

\end{abstract}


\section{Background and Introduction}
        The quest to measure the speed of light has historically led to the development of innovative experimental methods and a deeper understanding of the nature of light. This report documents an experiment conducted using the rotating mirror technique to measure the speed of light. 

        \subsection{Historical Context}
        The measurement of the speed of light has been a fascination in physics for centuries. 
        The rotating mirror method, a significant leap in this quest, was first employed by Léon 
        Foucault in the mid-19th century. Foucault's method notably improved upon previous attempts 
        by using a rotating mirror to measure the time it took for light to travel a certain 
        distance. The accuracy he achieved was within 0.17\% of the speed later recognized as the 
        constant 'c'. Albert Michelson further refined this method, extending the path length 
        to up to 44 miles and improving the precision to 1.3 parts per million. His work laid 
        the foundation for the current experimental setups used in speed of light measurements
        
        \subsection{Modern Implications}
        Discuss the importance of the speed of light as a fundamental constant in the modern SI system of units, its redefinition in 1983, and the implications for modern physics.

        \subsection{Principles of the Rotating Mirror Method}
        Explain the theory behind the rotating mirror method of measuring the speed of light, detailing the optical setup and the relationship between beam displacement and mirror rotation frequency.
    
        \subsection{Gaussian Beam Optics}
        Introduce the concept of Gaussian beams and discuss the significance of beam waist, spot size, and Rayleigh length in the context of this experiment.


\section{Experimental Setup and Procedure}
        
\section{Results}

\section{Discussion and Conclusion}

        \subsection{Error Analysis}



\section{References}
    \begin{enumerate}
        \sloppy
        \item  
    \end{enumerate}

\end{document}
