\documentclass[12pt]{article}

\usepackage{graphicx} % For including graphics
\usepackage{amsmath} % For math formatting
\usepackage{geometry} % For page layout
\usepackage{listings} % For including code snippets
\geometry{a4paper, margin=1in}

\title{Chaot Dynamics}
\author{Enrique Rivera Jr. \\
        Physics Undergraduate, \\ 
        The University of Texas at Austin}
\date{\today}

\begin{document}
\maketitle

\begin{abstract}
        This paper presents an investigation into the relativistic effects on electrons emitted through 
        the decay of Sodium-22 (Na-22), Cobalt-60 (Co-60), and etc. By analyzing the energy and momentum of these electrons, 
        we compare experimental data with classical Newtonian and relativistic predictions to underscore 
        the necessity of relativistic considerations at high energies.
\end{abstract}

\section{Introduction}
    \subsection{Background and Theory}

        \subsubsection{Chaos Theory} 
        sample text 
        \subsubsection{Relativistic Kinematics}
        sample 

    \subsection{Purpose}
            This experiment is committed to the empirical confirmation of relativistic dynamics' theoretical projections and to delineate the boundaries of classical mechanics under the duress of high-energy conditions.


\section{Experimental Setup and Procedure}
        \subsection{Apparatus}
                \subsubsection{Equipment}
                The The following equipment (or equivalent) is needed for the experiment:

                        \begin{enumerate}
                                \item Gamma Source Kit Samples(Na-22, Co-60, etc.)
                                \item Scintillation Detector
                                \item Photomultiplier Tube
                                \item Amplifier ( Linear Amplifier (ORTEC 672))
                                \item Multichannel Analyzer
                                \item Computer
                        \end{enumerate}
                        
                \subsubsection{Scintillation Detector}
                        The scintillator is a device that detects radiation by converting the energy of incoming
                        photons into visible light. This light is then detected by a photomultiplier tube, which
                        amplifies the signal and converts it into an electrical pulse. The scintillator used in
                        this experiment is a sodium iodide (NaI) crystal, which is commonly used for detecting
                        gamma radiation. The crystal is coupled to a photomultiplier tube, which amplifies the
                        light signal and converts it into an electrical pulse


                \subsubsection{Photomultiplier Tube}
                        The photomultiplier tube is a device that converts the light signal from the scintillator
                        into an electrical pulse. It consists of a series of dynodes, which are metal electrodes
                        that are held at successively higher voltages. When a photon strikes the first dynode, it
                        releases an electron, which is then accelerated towards the next dynode. This process is
                        repeated at each dynode, resulting in a cascade of electrons that is amplified at each stage.
                        The final output is a large number of electrons, which is then converted into an electrical
                        pulse. This pulse is then passed to a electronic preamplifier to be then processed by the
                        data acquisition system, which records the number of pulses over a given time interval.
                        This allows us to measure the intensity of the radiation emitted by the samples.

                        
                \subsubsection{Schematic}
                        This is a schematic of the experimental setup, which can be seen below in figure 2: \\ \\
                        
                        
                        The figure was kept simple to illustrate the basic components of the setup. 
                        The scintillation detector is used to measure the energy of the emitted electrons. 
                        The pulses from the scintillation detector are then passed to a linear amplifier and then 
                        to a multichannel analyzer, which records the number of pulses over a given time interval. 
                        The data is then analyzed to determine the energy of the emitted electrons and the intensity 
                        of the radiation emitted by the samples using Maestro.


                        For the Python analysis, the data was read in using the Maestro software which was outputed as a 
                        spe file which was converted to a csv with a custom python program that also took into account 
                        the energy calibration. We then visualized and analyzed this data using the Pandas, Matplotlib, and Altir libraries. 
                        The data was then plotted to show gamma decay spectrum and the notable traits of the graph.



\section{Results}
        \subsection{Data and Analysis}

                \subsubsection{Na-22}
                The data for Na-22 was analyzed and the energy of the emitted electrons was determined. 
                The energy of the emitted electrons was then plotted as a function of their momentum, 
                as shown in figure 3. The data was then fitted to a linear function to determine the slope 
                of the line, which is the speed of the electrons. The speed of the electrons was then compared 
                to the speed of light to determine if the electrons were traveling at relativistic speeds.

                \subsubsection{Co-60}
                The data for Co-60 was analyzed and the energy of the emitted electrons was determined. 
                The energy of the emitted electrons was then plotted as a function of their momentum, 
                as shown in figure 4. The data was then fitted to a linear function to determine the slope 
                of the line, which is the speed of the electrons. The speed of the electrons was then compared 
                to the speed of light to determine if the electrons were traveling at relativistic speeds.

                \subsubsection{Compton Edge}
                The Compton edge was determined by analyzing the data for the Compton scattering of the electrons. 
                The energy of the electrons was then plotted as a function of their momentum, as shown in figure 6. 
                The data was then fitted to a sinusoidal function to determine the Compton edge, which is the maximum 
                energy of the electrons emitted by the sample. The Compton edge was then compared to the theoretical 
                value to determine if the electrons were traveling at relativistic speeds.

                \subsubsection{Comparison}
                The experimental data was then compared to the theoretical predictions for the energy and momentum of the 
                emitted electrons. The experimental data was found to be consistent with the relativistic predictions as shown in figure 7. 
                Confirming the validity of the relativistic equations for kinetic energy and momentum. This also highlights 
                the limitations of classical mechanics at high energies, and the necessity of relativistic considerations in such scenarios.
        





\section{Conclusion}
        So from the results we can see that the experimental data is consistent with the relativistic predictions. This confirms the validity of the relativistic equations for kinetic energy and momentum. This also highlights the limitations of classical mechanics at high energies, and the necessity of relativistic considerations in such scenarios.


\section{References}
    \begin{enumerate}
        \sloppy

        \item Greg O. Sitz. "PHY353L Spring 2024 Lecture 2" Lecture, Physics 353L, University of Texas, Austin, TX, January 29, 2024.
        \item Radiation Detection Module. (n.d.). Retrieved September 14, 2021, from https://www.amptek.com/products/radiation-detection-module/
        \item J. Higbie, Am. J. Phys. 42, 642 (1974), Undergraduate Relativity Experiment.
        \item The Science and Maths Zone. (2021, March 14). The Photoelectric Effect, Photons and Planck’s Equation. The Science and Maths Zone. https://thescienceandmathszone.com/the-photoelectric-effect-photons-and-plancks-equation/

    \end{enumerate}

\end{document}
