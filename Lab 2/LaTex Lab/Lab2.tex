\documentclass[12pt]{article}

\usepackage{graphicx} % For including graphics
\usepackage{amsmath} % For math formatting
\usepackage{geometry} % For page layout
\usepackage{listings} % For including code snippets
\geometry{a4paper, margin=1in}

\title{Effect of Lead Thickness on Radiation Detection}
\author{Enrique Rivera Jr. \\
        Physics Undergraduate, \\ 
        The University of Texas at Austin}
\date{\today}

\begin{document}
\maketitle

\begin{abstract}
    In our investigation, we examined how varying thicknesses of lead 
    influence the detection of radiation emitted by Cobalt-60, a widely utilized radioactive isotope. 
    Utilizing a scintillator, we quantified the radiation intensity across different lead shields. 
    Our observations revealed a predictable decline in radiation intensity with increased lead thickness, 
    aligning with theoretical expectations. Notably, this attenuation follows an exponential trend, 
    underscoring the complex dynamics of radiation shielding. These insights contribute valuable knowledge towards 
    optimizing radiation protection materials, emphasizing the exponential nature of radiation attenuation 
    through lead. This understanding is crucial for designing more effective shielding strategies in 
    both medical and industrial applications, enhancing safety protocols against radioactive exposure.
\end{abstract}


\end{document}
